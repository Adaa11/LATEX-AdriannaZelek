\documentclass[a4paper,11pt]{article}
\usepackage{polski}
\usepackage[utf8]{inputenc}
\usepackage{latexsym}
\usepackage{color}
\usepackage{graphicx} 
\usepackage{geometry}
\usepackage{url}
\usepackage{hyperref}
\usepackage{enumerate}
\author{Adrianna Zelek}
\title{BASH $\heartsuit$}
\frenchspacing
\begin{document}
\newgeometry{tmargin=3cm, bmargin=3cm, lmargin=3cm, rmargin=3cm}
\maketitle
\newpage

\tableofcontents

\newpage
\section{Wiadomości podstawowe} 

\textcolor{blue}{BASH} - powłoka systemowa UNIX napisana dla projektu GNU. Program jest rozprowadzany na licencji GPL.


Bash to jedna z najpopularniejszych powłok systemów uniksowych. Jest domyślną powłoką w większości dystrybucji systemu GNU/Linux oraz w systemie OS X od wersji 10.3, istnieją także wersje dla większości systemów uniksowych. Bash jest także domyślną powłoką w środowisku Cygwin dla systemów Win32.


Bash pozwala na pracę w trybie konwersacyjnym i w trybie wsadowym. Język basha umożliwia definiowanie aliasów, funkcji, zawiera konstrukcje sterujące przepływem (if, while, for, ...). Powłoka systemowa zachowuje historię wykonywanych poleceń i zapisuje ją domyślnie w pliku .bash-history w katalogu domowym użytkownika.

Nazwa jest akronimem od Bourne-Again Shell (angielska gra słów: fonetycznie brzmi tak samo, jak born again shell, czyli odrodzona powłoka). Wywodzi się od powłoki Bourne'a sh, która była jedną z pierwszych i najważniejszych powłok systemu UNIX oraz zawiera pomysły zawarte w powłokach Korna i csh. Bash był pisany głównie przez Briana Foksa i Cheta Rameya w 1987. ~\cite{pa}

\subsection{Co to jest skrypt?} 
\textbf{Skrypt} to niekompilowany tekstowy plik wykonywalny, zawierający jakieś
polecenia systemowe oraz polecenia sterujące jego wykonaniem (instrukcje,
pętle itp.). Skrypt wykonywany jest tylko i wyłącznie przez \textit{interpreter} (tutaj /bin/bash), który tłumaczy polecenia zawarte w skrypcie na język zrozumiały dla procesora.

\subsection{Hello world} 
Nasz pierwszy skrypt:\\
\textit{\#!/bin/bash \\
\# Tu jest komentarz.  \\
echo "Hello world" \\ }
\# - oznacza komentarz, wszystko co znajduje się za nim w tej samej linii, jest
pomijane przez interpreter. \\
Pierwsza linia skryptu zaczynająca się od znaków: \# ! ma szczególne
znaczenie - wskazuje na rodzaj shella w jakim skrypt ma być wykonany, tutaj
skrypt zawsze będzie wykonywany przez interpreter poleceń /bin/bash,
niezależnie od tego jakiego rodzaju powłoki w danej chwili używamy. \ref{rys:h:obrazek1}\\
\begin{figure}[h]
\centering
\includegraphics[width=12cm]h
\caption{Wykonanie skryptu hello\_world.sh}
\label{fig:obrazek h} \label{rys:h:obrazek1}
\end{figure}
\newpage
\section{Zmienne} 
\subsection{Zmienne programowe} 
Zmienne programowe (użytkownika) (ang. program variables) - są to zmienne po prostu wykorzystywane wewnątrz skryptu, które definiuje użytkownik. Nie są dostępne dla innych programów, skryptów i powłok potomnych. Ich nazwy nie mogą się zaczynać od cyfry i pisane są zwyczajowo małymi literami. Czasami stosuje się podkreślenia.


Zmienne tego typu inicjujemy w następujący sposób: \\
\colorbox{yellow}{nazwa\_zmiennej="wartość"} \\
a korzystamy z nich, poprzedzając ich wartość znakiem dolara (\$). \\
\colorbox{yellow}{echo \$nazwa\_zmiennej}

\subsection{Zmienne systemowe}
Zmienne systemowe, często nazwane także zmiennymi środowiskowymi (ang.environment variables)- są zdefiniowane i przechowywane w środowisku nadrzędnym (powłoce z której uruchamiane są Twoje skrypty). Ich nazwy są zwyczajowo tworzone z dużych liter i mogą być testowane poleceniem set. Definiują środowisko użytkownika, dostępne dla wszystkich procesów potomnych. \cite{p} \\
Przykłady:
\begin{itemize}
 \item \$HOME ścieżka do twojego katalogu domowego
 \item \$USER twój login 
 \item \$HOSTNAME nazwa twojego hosta
 \item \$OSTYPE rodzaj systemu operacyjnego
 \item \$UID Unique User ID - unikalny numer użytkownika
 \item \$TERM Typ terminala z jakiego korzystamy \ldots
\end{itemize}
 
\subsection{Zmienne tablicowe}
BASH pozwala na stosowanie zmiennych tablicowych jednowymiarowych.
Tablica to zmienna która przechowuje listę jakichś wartości (rozdzielonych
spacjami). W BASH'u nie ma maksymalnego rozmiaru tablic. Kolejne wartości zmiennej tablicowej indeksowane są przy pomocy liczb całkowitych, zaczynając od 0. \cite{s}\\
\textit{Składnia} \\
zmienna=(wartość1 wartość2 wartość3 wartośćn) \\
Przykład:  \\
\textit{ \#!/bin/bash \\
tablica=(element1 element2 element3)\\
echo \${tablica[0]}\\
echo \${tablica[1]}\\
echo \${tablica[2]}\\ }
\subsection{Zmienne specjalne}
\textbf{\$0} nazwa bieżącego skryptu lub powłoki \\
Przykład:\\
\textit{\#!/bin/bash\\
echo "\$0"} \\
\textcolor{blue}{Pokaże nazwę naszego skryptu}. \\ \\
\textbf{\$1..\$9} Parametry przekazywane do skryptu (wyjątek, użytkownik może modyfikować ten rodzaj \$-ych specjalnych.\ref{rys:p:obrazek2}\\
\textit{\#!/bin/bash 
echo "\$1 \$2 \$3" \\
}
Jeśli wywołamy skrypt z parametrami to przypisane zostaną zmiennym: od \$1 do \$9. \\ \\
\textbf{\$@} \\
\textcolor{blue}{Pokaże wszystkie parametry przekzywane do skryptu (też wyjątek), równoważne \$1 \$2 \$3..., jeśli nie podane są żadne parametry \$@ interpretowana jest jako nic.} \\
Przykład: \\
\textit{
\#!/bin/bash \\
echo "Skrypt uruchomiono z parametrami: \$@"
} \\ \\
skrypt.sh {obrazek2}\\ 
\textit{\#!/bin/bash \\
echo "\$0" \\
echo "\$1 \$2 \$3" \\
echo "Skrypt uruchomiono z parametrami: \$@"} \\
\begin{figure}[h]
\centering
\includegraphics[width=12cm]p
\caption{Wykonanie skryptu skrypt.sh}\label{rys:p:obrazek2}
\label{fig:obrazek p}
\end{figure}

\section{Operacje Arytmetyczne}
\subsection{Dodawanie i odejmowanie}
\begin{enumerate}
  \item  Polecenie "let" \\
  \textit{ let result=123+4\\
   echo \$result }
  \item Operator [ ] \\
  \textit{ result=\$[ 5 + 5 ]\\
    echo \$resultt }
  \item  Polecenie "expr" \\
  \textit{result=`expr 5 + 5 ` \\
  echo \$result}
\end{enumerate}

\subsection{Potęgowanie i pierwiastkowanie}
Pierwiastkowanie $\sqrt{x}$ \\
\textit{echo "sqrt(100)" $|$ bc} \\
Potęgowanie $a^{n}$ \\
\textit{echo "sqrt(2 $\land$12)" $|$ bc} \\

\section{Dostęp do plików}
chmod (ang. change mode – zmiana atrybutu) – polecenie zmiany zezwoleń dostępu do plików w systemach uniksowych. \cite{c} \ref{tab:jeden} \\ \\
%\textit{Tabela z interpretacją kodów ósemkowych} \\
\begin{table}[h]
\centering
\begin{tabular}{|r|l|l|} \hline
\textbf{Cyfra} & \textbf{Prawa} & \textbf{Litera}\\ \hline
0 &	Brak praw &  \\ \hline
1 &	Wykonywanie & x\\ \hline 
2 &	Pisanie & w\\ \hline
4 & Czytanie & r \\ \hline
\end{tabular} 
\caption{Tabela z interpretacją kodów ósemkowych}\label{tab:jeden}
\end{table} \\

Przykłady użycia: 
\begin{itemize}
\item \$ chmod a+w plik.txt — nadaje wszystkim uprawnienia do zmiany 'plik.txt',
\item \$ chmod o-x plik.txt — usuwa możliwość wykonywania 'plik.txt' przez pozostałych użytkowników,
\item \$ chmod go=rx plik.txt — grupa oraz pozostali użytkownicy będą mogli tylko czytać i wykonywać.
\item \$ chmod -R 777 /home/user — wszyscy będą mogli zmieniać zawartość katalogu /home/user oraz jego podkatalogów, jak też czytać go i wykonywać zawarte w nim pliki
\end{itemize}

\section{Trudne wzory}

Całka \ref{equ:calka} \\
$\lim_{n \to \infty}
\sum_{k=1}^n \frac{1}{k^2}
= \frac{\pi^2}{6}
$
\\ \\
$
\sum_{i=1}^{n} \quad
\int_{0}^{\frac{\pi}{2}}
\label{equ:calka}
$ \\ \\ 
$y = \left\{ \begin{array}{ll}
a & \textrm{gdy $d>c$}\\
b+x & \textrm{gdy $d=c$}\\
l & \textrm{gdy $ d < c $}
\end{array} \right.
$

\newpage


\begin{thebibliography}{99}
\bibitem{pa}
 \href{http://pl.wikipedia.org/wiki/Bash}{Wikipedia bash} 
\bibitem{p}
 \href{http://bash.0x1fff.com/zmienne-systemowe}{www.bash.0x1fff.com}
 \bibitem{s} 
 prezentacja - Środowisko programisty - BASH  Emilia Lubecka
\bibitem{c} 
 \href{http://pl.wikipedia.org/wiki/Chmod}{Wikipedia Chmod} 
\end{thebibliography}

\end{document}